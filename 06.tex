\documentclass[../Chapter.tex]{subfiles}

\begin{document}

\chapter{Quadratic Gauss Sums}

\subsection*{Exercise 6.8}

We know $1+\omega+\omega^2=0$. So $(2\omega+1)^2=4\omega^2+4\omega+1=4(1+\omega+\omega^2)-3=-3$. Set $\tau:=2\omega+1$, then $\tau^2=-3$.

Let $p\neq 3$ be an odd prime. Working with congruences in the ring of algebraic integers, we have $\tau^{p-1}=(\tau^2)^{(p-1)/2}=(-3)^{(p-1)/2}\equiv (-3/p) \pmod{p}$. So $\tau^p\equiv (-3/p)\tau \pmod{p}$. On the other hand,
$$\tau^p = (2\omega+1)^p \equiv 2^p\omega^p+1 \equiv
\begin{cases}
2\omega+1 = \tau \pmod{p}, & p\equiv 1\pmod{3} \\
2\omega^2+1 = -\tau \pmod{p}, & p\equiv 2\pmod{3}
\end{cases}$$
Combining these two, multiplying both sides by $\tau$ and cancelling out some terms, we get $$\left(\dfrac{-3}{p}\right) \equiv
\begin{cases}
1 \pmod{p}, & p\equiv 1\pmod{3} \\
-1 \pmod{p}, & p\equiv 2\pmod{3}
\end{cases}$$
This congruence equation implies the quality.

\subsection*{Exercise 6.10}

By Prop 6.3.1 (p. 71), $\sum_{a=1}^{p-1} g_a = \sum_{a=1}^{p-1} (a/p)g = g\sum_{a=1}^{p-1} (a/p)=0$.

\subsection*{Exercise 6.11}

First note that $\sum_t (1+(t/p))\zeta^t = \sum_t \zeta^t + \sum_t (t/p)\zeta^t =g$. On the other hand, $\sum_t (1+(t/p))\zeta^t =1 + \sum_{t\neq 0 \text{ is a square}} 2\zeta^t = \zeta^{0^2} + 2\sum_{t\neq 0}\zeta^{t^2}/2 = \sum_t \zeta^{t^2}$. Hence we have $g=\sum_t \zeta^{t^2}$.

\subsection*{Exercise 6.13}

Fix $t_0\in\ZZ$. Set $s_0\equiv t_0 \pmod{p}$ where $0\leq s_0\leq p-1$. This means $f(s_0)=f(t_0)$. Using Ex 6.12, we have
\begin{align*}
\sum_a \hat{f}(a)\psi_a(t_0) &= \sum_a\left(p^{-1}\sum_s f(s)\psi_{-a}(s)\right)\cdot\psi_a(t_0) \\
&= p^{-1} \sum_s f(s)\cdot\left(\sum_a \psi_{-a}(s)\psi_a(t_0)\right) \\
&\overset{(a)}{=} p^{-1} \sum_s f(s)\cdot\left(\sum_a \psi_a(t_0-s)\right) \\
&\overset{(b)}{=} \sum_s f(s)\delta(t_0,s) = f(s_0)=f(t_0)
\end{align*}
This holds for each $t_0\in\ZZ$, so $f(t)=\sum_a \hat{f}(a)\psi_a(t)$.

\subsection*{Exercise 6.15}

Summing up the equation $(t/p)g=g_t$, $\lhs$ gives us $|\sum_{t=m}^n (t/p)g|=|g||\sum_{t=m}^n (t/p)|=\sqrt{p}\cdot|\sum_{t=m}^n (t/p)|$. For the $\rhs$, we have
\begin{align*}
\left|\sum_{t=m}^n g_t\right| &= \left|\sum_{t=m}^n \sum_x \left(\dfrac{x}{p}\right)\zeta^{xt}\right| = \left| \sum_{x=1}^{p-1} \left(\dfrac{x}{p}\right)\cdot\left(\sum_{t=m}^n \zeta^{xt}\right) \right| \\
&= \left|\sum_{x=1}^{p-1} \left(\dfrac{x}{p}\right)\zeta^{mx}\cdot\frac{\zeta^{(n-m+1)x}-1}{\zeta^x-1}\right| \leq \sum_{x=1}^{p-1} \left|\frac{\zeta^{(n-m+1)x}-1}{\zeta^x-1}\right|
\end{align*}
Note that $\zeta^x-1=e^{2\pi ix/p}-1=e^{\pi ix/p}(e^{\pi ix/p}-e^{-\pi ix/p})=e^{\pi ix/p}\sin(\pi x/p)\cdot2i$. Similarly, $\zeta^{(n-m+1)x}-1=e^{\pi i(n-m+1)x/p}\sin(\pi(n-m+1)x/p)\cdot2i$. So we have
\begin{align*}
\left|\sum_{t=m}^n g_t\right| \leq \sum_{x=1}^{p-1} \left|\frac{e^{\pi i(n-m+1)x/p}\sin(\pi(n-m+1)x/p)\cdot2i}{e^{\pi ix/p}\sin(\pi x/p)\cdot2i}\right| \leq \sum_{x=1}^{p-1} \left|\frac{1}{\sin(\pi x/p)}\right|
\end{align*}

For $x=1,\ldots,(p-1)/2$, we have $\pi x/p<\pi/2$. So $1/\sin(\pi x/p)\leq1/((2/\pi)(\pi x/p))=p/2x$. This implies $$\sum_{x=1}^{(p-1)/2} \frac{1}{\sin(\pi x/p)} \leq \sum_{x=1}^{(p-1)/2} \frac{p}{2x}$$
And for $x=(p+1)/2,\ldots,p-1$, we have $\pi/2<\pi x/p<\pi$. Since $\sin(\pi x/p)=\sin(\pi-\pi x/p)$ and $\pi-\pi x/p=\pi/p,2\pi/p,\ldots,(p-1)\pi/2p$ when $x$ varies from $p-1$ to $(p+1)/2$, so $$\sum_{x=(p+1)/2}^{p-1} \frac{1}{\sin(\pi x/p)}=\sum_{x=(p+1)/2}^{p-1} \frac{1}{\sin(\pi-\pi x/p)} = \sum_{x=1}^{(p-1)/2} \frac{1}{\sin(\pi x/p)}$$
Combining these two, we obtain
\begin{align*}
\left|\sum_{t=m}^n g_t\right| &\leq \sum_{x=1}^{p-1} \left|\frac{1}{\sin(\pi x/p)}\right| = \sum_{x=1}^{p-1} \frac{1}{\sin(\pi x/p)} = 2\sum_{x=1}^{(p-1)/2} \frac{1}{\sin(\pi x/p)} \\
&\leq 2\sum_{x=1}^{(p-1)/2} \frac{p}{2x} = p\sum_{x=1}^{(p-1)/2} \frac{1}{x} \leq p\sum_{x=1}^{(p-1)/2} \ln\left(\frac{2x+1}{2x-1}\right) = p\ln p
\end{align*}
The last inequality is just simple Calculus.

Combining the $\lhs$ and $\rhs$, we obtain $\sqrt{p}\cdot|\sum_{t=m}^n (t/p)|\leq p\ln p$. The result now follows easily by dividing both sides by $\sqrt{p}$. Hence, we have $|\sum_{t=m}^n (t/p)|\leq \sqrt{p}\ln p$.


\end{document}