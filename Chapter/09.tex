\documentclass[../I&R.tex]{subfiles}

\begin{document}

\chapter{Cubic and Biquadratic Reciprocity}

\subsection*{Exercise 9.1}

Set $\lambda:=1-\omega$ and $D:=\ZZ[\omega]$. By Prop 9.2.1 (p. 111), $\#(D/\lambda D)=3$. So it's sufficient to show that the residue classes of $0,1,-1$ are distinct in $D/\lambda D$. Obviously, we have $0\not\equiv1\pmod{\lambda}$ and $0\not\equiv-1\pmod{\lambda}$. And note that $1\equiv-1\pmod{\lambda} \iff \lambda\mid2$. A simple calculation shows that this is not the case. This completes the proof.

Note that $\omega\equiv1\pmod{\lambda}$. And $\omega^2=-1-\omega=-2+(1-\omega)\equiv-2=1-3\equiv1\pmod{\lambda}$. So $1\equiv\omega\equiv\omega^2\pmod{\lambda}$. (See also: page 112. Notice that $N(\lambda)=3$.)

\subsection*{Exercise 9.2}

This follows easily by Prop 9.1.1 (p. 109), Prop 9.3.5 (p. 113), and the fact that $D$ is a UFD. See \nameref{Bonus 9.1} for the classification of primes.

\subsection*{Exercise 9.3}

$\chi_\gamma(\omega)=\omega^{(N(\gamma)-1)/3}$. Note that $N(\gamma)=(3m-1)^2-(3m-1)(3n)+(3n)^2\equiv3(m+n)+1\pmod{9}$. So write $N(\gamma)=9k+3(m+n)+1$, then $\chi_\gamma(\omega)=\omega^{(N(\gamma)-1)/3}=\omega^{3k+m+n}=\omega^{m+n}$.

\subsection*{Exercise 9.4}

First note that $\gamma=(3m-1)+3n\omega=-1+3(m+n)-3n\lambda$ and so $\gamma\equiv-1+3(m+n) \pmod{3\lambda}$. Also, we have $D/\lambda D\simeq \ZZ/3\ZZ$. And by Ex 9.1, the residue classes of $-1,0,1$ in $D/\lambda D$ correspond to the residue classes of $-1,0,1$ in $\ZZ/3\ZZ$. (See the beginning of Sec 4 (p.115).)

Now, $\gamma\equiv-1+3(m+n)\equiv8\pmod{3\lambda} \iff m+n\equiv3\equiv0\pmod{\lambda}$. (Recall that $3=-\omega^2\lambda^2$.) And by the above isomorphism, this is equivalent to $m+n\equiv0\pmod{3}$. So in this case $\chi_\gamma(\omega)=1$ by Ex 9.3. The other two cases are similar, we skip the details.

In particular, if $q\equiv2\pmod{3}$ is a rational prime. Write $q=3m-1$, then $\chi_\gamma(\omega)=\omega^m$. Note that $m\equiv0,1,2\pmod{3}$ correspond to $q\equiv8,2,5\pmod{9}$, respectively. This completes the proof.

\subsection*{Exercise 9.5}

$\chi_\gamma(3)=\chi_\gamma(-\omega^2\lambda^2)=\chi_\gamma(-1)\chi_\gamma(\omega)^2\chi_\gamma(\lambda)^2=1\cdot\omega^{2m+2n}\cdot\omega^{4m}=\omega^{2n}$.

\subsection*{Exercise 9.6}

We use the same notation as in Ex 9.3. And recall that $\chi_\gamma(\lambda)=\omega^{2m}$.

(a) $\chi_\gamma(\lambda)=1 \iff 2m\equiv0\pmod{3} \iff m\equiv0\pmod{3} \iff a\equiv8\pmod{9}$. So in this case we have $\gamma\equiv8,8+3\omega,8+6\omega\pmod{9}$.

(a) $\chi_\gamma(\lambda)=\omega \iff 2m\equiv1\pmod{3} \iff m\equiv2\pmod{3} \iff a\equiv5\pmod{9}$. So in this case we have $\gamma\equiv5,5+3\omega,5+6\omega\pmod{9}$.

(a) $\chi_\gamma(\lambda)=\omega^2 \iff 2m\equiv2\pmod{3} \iff m\equiv1\pmod{3} \iff a\equiv2\pmod{9}$. So in this case we have $\gamma\equiv2,2+3\omega,2+6\omega\pmod{9}$.

\subsection*{Exercise 9.7}

$(1-2\omega)(\omega)=2+3\omega$ is primary. $-7-3\omega$ is itself primary. $(3-\omega)(\omega^2)=-4-3\omega$ is primary.

\subsection*{Exercise 9.8}

We use Prop 9.1.4 (p. 110).

From $x^3-1=(x-1)(x-\omega)(x-\omega^2)$ we have $7=(2-\omega)(2-\omega^2)=(2-\omega)(3+\omega)$. This gives us $21=3\cdot7=-\omega^2\lambda^2(2-\omega)(3+\omega)$.

$45=5\cdot3^2=5\omega\lambda^4$. Lastly, write $143=11\cdot13$. Set $\pi=a+b\omega$ and $13=\pi\ovl{\pi}$, then $13=N(\pi)=a^2-ab+b^2$. Take $a=3,b=-1$. So $13=11\cdot13=11(3-\omega)(4+\omega)$.

\subsection*{Exercise 9.9}

We assume $\pi\not\sim\lambda$ because the cubes in $(D/\lambda D)^\times$ is clear. (See Ex 9.1.) From Prop 7.1.2 (p. 80) we have $x^3=\ovl{\alpha}$ has a solution in $(D/\pi D)^\times \iff \ovl{\alpha}^{(N(\pi)-1)/3}=\ovl{1}$ in $(D/\pi D)^\times \iff \alpha^{(N(\pi)-1)/3}\equiv1\pmod{\pi}$. Consequently, we have $\#(\text{cubes in }(D/\pi D)^\times)=\#(\text{solutions to the equation } x^{(N(\pi)-1)/3}\equiv1\pmod{\pi})=(N(\pi)-1)/3$.

\subsection*{Exercise 9.10}

Since $D/5D$ is a finite field with $N(5)=25$ elements, we have $\alpha^{24}=1$ for each $\alpha\in(D/5D)^\times$. So $x^{24}-1=\prod_{\alpha\neq0} (x-\alpha)$.

\subsection*{Exercise 9.11}

By Ex 9.9, there are $(N(5)-1)/3=8$ cubes.

\subsection*{Exercise 9.12}

Note that $a+b\omega\equiv1\pmod{5} \iff a\equiv1\pmod{5}$ and $b\equiv0\pmod{5}$. Also, we know $(\omega\lambda)^2=-3$. From this it's easy to see that $\ord((\omega\lambda)^2)=4$. Consequently, $\ord(\omega\lambda)=8$.

For $\omega^2\lambda=\omega\cdot\omega\lambda$, since $\gcd(\ord(\omega),\ord(\omega\lambda))=\gcd(3,8)=1$, we have $\ord(\omega^2\lambda)=\ord(\omega)\cdot\ord(\lambda)=3\cdot8=24$.

\subsection*{Exercise 9.13}

By Ex 9.9 and Ex 9.11, it's sufficient to check the eight numbers all satisfy $x^8\equiv1\pmod{5}$. For $x=1,2,3,4$ this is easy. And for $x=1+2\omega$, since $x^2=-3$ we have $x^8=81\equiv1\pmod{5}$. So the desired condition holds. Note that this implies $2(1+2\omega),3(1+2\omega),4(1+2\omega)$ are all cubes. They are congruent to $2+4\omega,3+\omega,4+3\omega \pmod{5}$, respectively. So we are done.

\subsection*{Exercise 9.14}

Since $x^3\equiv5\pmod{\pi}$ is solvable iff $x^3\equiv5\pmod{\pi'}$ is solvable for $\pi'\sim\pi$, we may assume $\pi$ is primary. By the cubic reciprocity law (p. 114) we have $\chi_\pi(5)=\chi_5(\pi)$. So $x^3\equiv5\pmod{\pi}$ is solvable iff $x^3\equiv\pi\pmod{5}$ is solvable. And from Ex 9.13 this is equivalent to say $\pi\equiv1,2,3,4,1+2\omega,2+4\omega,3+\omega,4+3\omega \pmod{5}$.

\subsection*{Exercise 9.15}

$(\Rightarrow)$ Suppose $x^3\equiv a\pmod{p}$ is solvable in $\ZZ$, then there exists $h\in\ZZ$ s.t. $h^3\equiv a\pmod{p} \implies p\mid h^3-a \implies \pi\mid h^3-a \implies h^3\equiv a\pmod{\pi}$. So $\chi_\pi(a)=1$.

$(\Leftarrow)$ Suppose $\chi_\pi(a)=1$, then $x^3\equiv a\pmod{\pi}$ is solvable. Since the elements in $D/\pi D$ are represented by rational integers (recall that $D/\pi D\simeq \ZZ/p\ZZ$), there exists $h\in\ZZ$ s.t. $h^3\equiv a\pmod{\pi}$. This means $\pi$ divides $h^3-a$ and thus so is $\ovl{\pi}$. Hence, $p\mid (h^3-a)^2$ and so $p\mid h^3-a$. This shows $x^3\equiv a\pmod{p}$ is solvable in $\ZZ$.

\subsection*{Exercise 9.16}

(Correction: $x^3\equiv2-3\omega\pmod{11}$ {\color{red}is} solvable as we will see later. The given hint is totally nonsense!)

Let $\pi=2-3\omega$, a primary prime, and $p=\pi\ovl{\pi}=19$. By the cubic reciprocity law (p. 114), we have $\chi_{11}(\pi)=\chi_\pi(11)$. So $x^3\equiv\pi\pmod{11}$ is solvable iff $x^3\equiv11\pmod{\pi}$ is solvable, i.e., $\chi_\pi(11)=1$. By Ex 9.15, this is equivalent to say $x^3\equiv11\pmod{19}$ is solvable in $\ZZ$. A simple calculation shows that this is indeed the case because for instance $5$ is one of the solutions. Hence, $x^3\equiv\pi=2-3\omega\pmod{11}$ is solvable.

With the help of computer, it's easy to verify this result. For instance, take $x=1-3\omega$, then $x^3=-53-36\omega\equiv2-3\omega\pmod{11}$.

\begin{comment}

\subsection*{Exercise 9.17}

By Ex 9.2, we may write $\gamma=\pm\omega^b\lambda^c\gamma_1\cdots\gamma_t$ where $\gamma_i$ is a primary prime for all $i$. So $2\equiv\gamma\equiv\pm\omega^b\lambda^c2^t\pmod{3}$. This implies $\omega^b\lambda^c\equiv1,2\pmod{3}$. Note that since $\lambda^2=-3\omega\equiv0\pmod{3}$, we have $c=0,1$. So we only need to check six cases, namely, $(b,c)=(0,0),(1,0),(2,0),(0,1),(1,1),(2,1)$. These correspond to $\omega^b\lambda^c=1,\omega,\omega^2,\lambda,\omega\lambda,\omega^2\lambda$, respectively.

Since $a+b\omega\equiv1,2\pmod{3}$ is equivalent to $a\equiv1,2\pmod{3}$ and $b\equiv0\pmod{3}$, it's easy to check that the only possible case is $(b,c)=(0,0)$. This implies $\gamma=\pm\gamma_1\cdots\gamma_t$ and so we are done.

\subsection*{Exercise 9.18}

Note that if $\alpha\equiv\beta\pmod{\gamma}$, then $\alpha\equiv\beta\pmod{\gamma_i}$ for all $i=1,\ldots,t$. So $\chi_\gamma(\alpha) \overset{\df}{=} \prod_{i=1}^t \chi_{\gamma_i}(\alpha) = \prod_{i=1}^t \chi_{\gamma_i}(\beta) \overset{\df}{=} \chi_\gamma(\beta)$. Also, $\chi_\gamma(\alpha\beta) \overset{\df}{=} \prod_{i=1}^t \chi_{\gamma_i}(\alpha\beta) = \prod_{i=1}^t \chi_{\gamma_i}(\alpha)\chi_{\gamma_i}(\beta) = \prod_{i=1}^t \chi_{\gamma_i}(\alpha)\prod_{i=1}^t \chi_{\gamma_i}(\beta) \overset{\df}{=} \chi_\gamma(\alpha)\chi_\gamma(\beta)$.

If $\rho=\pm\rho_1\cdots\rho_s$ is a primary decomposition of $\rho$, then $-\rho\gamma=\pm(\rho_1\cdots\rho_s\gamma_1\cdots\gamma_t)$ is a primary decomposition of $-\rho\gamma$. So $\chi_\rho(\alpha)\chi_\gamma(\alpha) \overset{\df}{=} \prod_{j=1}^s \chi_{\rho_j}(\alpha)\prod_{i=1}^t \chi_{\gamma_i}(\alpha) \overset{\df}{=} \chi_{-\rho\gamma}(\alpha)$.

\end{comment}

\subsection*{Exercise 9.43}

Let $f(x)=x^3-3px-Ap$, then $f'(x)=3x^2-3p=0$ at $x=\pm\sqrt{p}$. It follows that $f(x)$ has local maximum at $x=-\sqrt{p}$ and local minimum at $x=\sqrt{p}$. Note also that $|2\re(\omega^kg(\chi_\pi))|\leq2|g(\chi_\pi)|=2\sqrt{p}$ for $k=0,1,2$, so the roots of $f(x)$ lie in the interval $(-2\sqrt{p},2\sqrt{p})$. These two observations imply the result. (One can easily check that all the roots don't occur at the endpoints. For example, if $f(2\sqrt{p})=0$, then $p=A^2/4$ and so $B=0$. This implies $b=0$ and $A=2a$. (See the Cor of Prop 8.3.4 (p. 97).) So $p=a^2$, which is absurd.)

\subsection*{Bonus 9.1} \label{Bonus 9.1}

Let's classify all prime elements in $D=\ZZ[\omega]$. (See p. 110.) By Prop 9.1.2, it's enough to check $\pi\in D$ with $N(\pi)=p,p^2$ where $p$ is a rational prime. For $N(\pi)=p$, by Prop 9.1.3 we know $\pi$ is a prime. We now claim that $N(\pi)\equiv0,1\pmod{3}$. If $N(\pi)=\pi\cdot\ovl{\pi}=p\equiv2\pmod{3}$, then by Prop 9.1.4 we have $p$ is a prime in $D$, which is impossible because we know $\ovl{\pi}$ is also a prime.

For $N(\pi)=p^2$, we know from Prop 9.1.2 that $\pi\sim p$. So by Prop 9.1.4, $\pi$ is a prime in $D \iff p$ is a prime in $D\iff p\equiv 2\pmod{3}$. This means $\pi$ is associated to a rational prime $p\equiv 2\pmod{3}$.

Hence, $\pi\in D$ is a prime iff $N(\pi)=p$, a rational prime (in this case we have $N(\pi)\equiv1\pmod{3}$ or $p=3$), or $\pi$ is associated to a rational prime $p\equiv 2\pmod{3}$. Note that these two are mutually exclusive.

\subsection*{Bonus 9.2}

There's another viewpoint of Prop 9.2.1 (p. 111) by considering the splitting of prime $p\ZZ$ in $D$. Note that $N(\pi)=p,p^2$ implies $p\ZZ$ is a prime ideal lying under $\pi D$. Let $f$ be the corresponding inertial degree. We first assume $\pi\not\sim1-\omega$. In this case we have $N(\pi)\neq3$. And by \nameref{Bonus 9.1} there are two cases we have to consider.

If $N(\pi)=p\equiv1\pmod{3}$, then $pD=\pi\ovl{\pi}D=\pi D\cdot\ovl{\pi}D$. These two are different prime ideals because $\pi\not\sim\ovl{\pi}$. So we have $f=1$ and hence $\#(D/\pi D)=(\ZZ/p\ZZ)^f=p=N(\pi)$.

If $\pi$ is associated to a rational prime $p\equiv 2\pmod{3}$, then $pD=\pi D$, i.e., $p$ is inert in $D$. So we have $f=2$ and hence $\#(D/\pi D)=(\ZZ/p\ZZ)^f=p^2=N(\pi)$.

Now suppose $\pi\sim1-\omega$, then by Prop 9.1.4, $3D=(1-\omega)^2$. So we have $f=1$ and hence $\#(D/\pi D)=(\ZZ/3\ZZ)^f=3=N(\pi)$.

\end{document}